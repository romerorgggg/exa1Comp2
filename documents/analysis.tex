\documentclass[11pt,a4paper]{article}
\usepackage[utf8]{inputenc}
\usepackage{amsmath, amsfonts, amssymb, graphicx}
\usepackage[left=2.5cm,right=2.5cm,top=2.5cm,bottom=2.5cm]{geometry}
\usepackage{hyperref}

\title{Análisis de Péndulos Acoplados, Amortiguados y Forzados}
\author{Tu Nombre Aquí}
\date{\today}

\begin{document}
\maketitle

\begin{abstract}
    Este documento presenta el análisis de un sistema de dos péndulos acoplados con una interacción no lineal cuadrática, sujetos a fuerzas de amortiguamiento y a un forzamiento externo periódico. Se estudian fenómenos como la sincronización, la resonancia y la transición al caos.
\end{abstract}

\section{Modelo Físico}
El sistema está descrito por las siguientes ecuaciones diferenciales:
\begin{align}
    \ddot{\theta}_1 + \gamma \dot{\theta}_1 + \frac{g}{l}\sin(\theta_1) + \kappa(\theta_1 - \theta_2)^2 &= F_0 \cos(\omega t), \\
    \ddot{\theta}_2 + \gamma \dot{\theta}_2 + \frac{g}{l}\sin(\theta_2) + \kappa(\theta_2 - \theta_1)^2 &= F_0 \cos(\omega t).
\end{align}
Donde $\gamma$ es el coeficiente de amortiguamiento, y $F_0 \cos(\omega t)$ representa la fuerza de excitación externa.

\section{Análisis del Sistema}
Debido a los términos de amortiguamiento y forzamiento, la energía total del sistema no se conserva. El comportamiento a largo plazo puede tender a un estado estacionario, un ciclo límite o un atractor caótico, dependiendo de los parámetros.

\subsection{Comportamiento Energético}
La figura \ref{fig:energy} muestra la evolución de las energías individuales y la energía total. Se puede observar cómo la energía total tiende a un valor de equilibrio dinámico donde la energía inyectada por el forzamiento compensa la disipada por el amortiguamiento.

\begin{figure}[h!]
    \centering
    \includegraphics[width=0.8\textwidth]{../results/energy.png}
    \caption{Evolución de la energía en el sistema no conservativo.}
    \label{fig:energy}
\end{figure}

\subsection{Atractores en el Espacio de Fase}
Para ciertos regímenes de parámetros (por ejemplo, valores altos de $F_0$), el sistema exhibe comportamiento caótico, lo que se manifiesta en la aparición de atractores extraños en el espacio de fase.

\begin{figure}[h!]
    \centering
    \includegraphics[width=\textwidth]{../results/phase_space.png}
    \caption{Diagramas de espacio de fase. La complejidad de la trayectoria indica la naturaleza de la dinámica (periódica, cuasiperiódica o caótica).}
    \label{fig:phase_space}
\end{figure}


\section{Conclusiones}
Resumen de los hallazgos principales. Discusión sobre cómo los parámetros $\gamma$, $F_0$ y $\omega$ controlan la transición entre diferentes regímenes dinámicos.

\end{document}